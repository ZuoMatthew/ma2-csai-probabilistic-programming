We have extended the pipeline with support for parameter learning. For this functionality, our program expects a file containing tunable probabilities and another file containing values for all probabilities (the ground truth). The ground truth is necessary for generation of interpretations. The amount of interpretations to be generated can be set as well. More information about this feature can be found in \texttt{README.MD}.

\section{Generated interpretations}
Four interpretations can be generated with the following command:
\begin{lstlisting}
python3 scripts/inference.py --problog_learn files/problog/cancer_learn.pl --problog_learn_truth files/problog/cancer.pl --learning_interpretations 4
\end{lstlisting}
Observations will be dropped with a probability of 30\% automatically and the resulting interpretations will be written to src/files/interpretations.txt.

Here is an example of generated interpretations with the command listed above:
\begin{lstlisting}
evidence(\+cancer).
evidence(\+xray("positive")).
evidence(\+dyspnoea).
evidence(\+pollution("high")).
evidence(\+smoker).
====================================================
evidence(smoker).
evidence(dyspnoea).
evidence(\+pollution("high")).
evidence(pollution("low")).
====================================================
evidence(xray("negative")).
evidence(\+dyspnoea).
evidence(\+pollution("high")).
evidence(\+smoker).
====================================================
evidence(smoker).
evidence(dyspnoea).
evidence(\+pollution("high")).
evidence(pollution("low")).
\end{lstlisting}

\section{Pipeline with interpretations}

\subsection{Parameters with 10 interpretations}
The following results were reached after 45 iterations. Our current implementation determines convergence is reached when all parameters change by less than 0.005.
\begin{lstlisting}
smoker              : 0.3757309044335211
pollution("low")    : 0.6944157672162523
pollution("high")   : 0.1
p_cancer_0          : 1.7136773738417463e-05
p_cancer_1          : 1.5696846453312893e-12
p_cancer_2          : 0.05290598927777933
p_cancer_3          : 0.07266786066467187
p_xray("positive")_0: 0.9581146900452404
p_xray("negative")_0: 0.25501789141868203
p_xray("positive")_1: 0.0
p_xray("negative")_1: 0.8571427918086106
p_dyspnoea_0        : 0.15338511324535153
p_dyspnoea_1        : 0.0
\end{lstlisting}

\subsection{Parameters with 100 interpretations}
The pipeline was terminated after 1.5 hours at 39 iterations.
\begin{lstlisting}
smoker              : 0.3295379131986182
pollution("low")    : 0.9041099224149188
pollution("high")   : 0.09589007758508108
p_cancer_0          : 0.1110126093038839
p_cancer_1          : 0.00048335141926639366
p_cancer_2          : 0.37732149668302517
p_cancer_3          : 0.1502500596905257
p_xray("positive")_0: 0.8927878392142493
p_xray("negative")_0: 0.11856852583630233
p_xray("positive")_1: 0.1408494511343412
p_xray("negative")_1: 0.8749301634679101
p_dyspnoea_0        : 0.09252183100200292
p_dyspnoea_1        : 0.30160276817000836
\end{lstlisting}

\subsection{Parameters with 1000 interpretations}
The pipeline was terminated after 1.5 hours at 3 iterations.
\begin{lstlisting}
probabilities after iteration 3:
smoker              : 0.29233350800326574
pollution("low")    : 0.9148055069262087
pollution("high")   : 0.08519449307379132
p_cancer_0          : 0.09403271863945559
p_cancer_1          : 0.025426090633783687
p_cancer_2          : 0.733743251011375
p_cancer_3          : 0.3007407777225963
p_xray("positive")_0: 0.3741545373041979
p_xray("negative")_0: 0.8205193870361593
p_xray("positive")_1: 0.22686346009032624
p_xray("negative")_1: 0.773832837404553
p_dyspnoea_0        : 0.23145524434562476
p_dyspnoea_1        : 0.33029011323407503
\end{lstlisting}

\section{Observations for different number of interpretations}
We notice that TODO \ldots
Voorspelling: Met minder interpretaties zijn de iteraties sneller (logisch, want minder queries per iteratie) dan met meer interpretaties. Met meer interpretaties gebeurt de convergence wel in minder iteraties omdat de EM dan beter werkt. Wel ook de total runtime er bij zetten.