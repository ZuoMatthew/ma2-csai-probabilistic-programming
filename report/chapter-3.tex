We have extended the pipeline with support for parameter learning. For this functionality, our program expects a file containing tunable probabilities and another file containing values for all probabilities (the ground truth). The ground truth is necessary for generation of interpretations. The amount of interpretations to be generated can be set as well. More information about this feature can be found in \texttt{README.MD}.

\section{Generated interpretations}
Four interpretations can be generated with the following command:
\begin{lstlisting}
python3 scripts/inference.py --problog_learn files/problog/cancer_learn.pl --problog_learn_truth files/problog/cancer.pl --learning_interpretations 4
\end{lstlisting}
Observations will be dropped with a probability of 30\% automatically and the resulting interpretations will be written to src/files/interpretations.txt.

Here is an example of generated interpretations with the command listed above:
\begin{lstlisting}
evidence(\+cancer).
evidence(\+xray("positive")).
evidence(\+dyspnoea).
evidence(\+pollution("high")).
evidence(\+smoker).
====================================================
evidence(smoker).
evidence(dyspnoea).
evidence(\+pollution("high")).
evidence(pollution("low")).
====================================================
evidence(xray("negative")).
evidence(\+dyspnoea).
evidence(\+pollution("high")).
evidence(\+smoker).
====================================================
evidence(smoker).
evidence(dyspnoea).
evidence(\+pollution("high")).
evidence(pollution("low")).
\end{lstlisting}

\section{Pipeline with interpretations}
\subsection{Parameters with 10 interpretations}

\subsection{Parameters with 100 interpretations}

\subsection{Parameters with 1000 interpretations}


\section{Observations for different number of interpretations}
We notice that TODO \ldots
Voorspelling: Met minder interpretaties zijn de iteraties sneller (logisch, want minder queries per iteratie) dan met meer interpretaties. Met meer interpretaties gebeurt de convergence wel in minder iteraties omdat de EM dan beter werkt. Wel ook de total runtime er bij zetten.